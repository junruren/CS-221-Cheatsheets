\subsection{Markov decision processes}

Find the maximum value policy by using MDPs that help us cope with randomness
and uncertainty, in order to find our way between an initial state and an end
state. (Goal: maximize rewards)

\subsubsection{Notations}

\Blue{Definition} based on Search definition:
\begin{itemize}
    \item $\text{Reward}(s,a,s')$ (replacing $\text{Cost}(s,a)$)
    \item $T(s,a,s')$: $\forall s,a, \sum_{s'\in S} T(s,a,s') \equiv 1$ (replacing $\text{Succ}(s,a)$)
    \item Discount: \emph{(living in the moment "greedy algo")} $0 \le \gamma \le 1$ \emph{(save for the future)}
\end{itemize}
Search is a special case of MDP where $T(s,a,s') = \begin{cases}
    1 & s' = \text{Succ}(s,a) \\
    0 & \text{otherwise}
\end{cases}$

\Blue{Policy} $\pi$ is a function that maps each state $s$ to an action $a \in
\text{Actions}(s)$. Following a policy yields a random path.

\Green{Utility (of a policy $\pi$ / path)} the (discounted) sum of the rewards on the
path, making it a random variable.
$\text{u}(\pi) = \sum_{i=0}^{\infty} \gamma^i \text{Reward}(s_i, \pi(s_i), s_i + 1)$

\Green{Value (of a policy $\pi$ at state $s_0$)} the expected utility received by
following policy $\pi$ from state $s$ over random paths; randomness comes from
$T(s,a,s')$ and possibly $\pi$. $V_\pi(s) = Q_\pi(s,\pi(s))$

\Blue{$Q$-value} the expected utility of a ``chance node'' (taking action $a$ from
state $s$ \underline{and then} following $\pi$).
$Q_\pi(s,a)
= \underbrace{\sum_{s'}T(s,a,s') \text{Reward}(s,a,s')}_{\text{exp'd rwd of taking (s, a)}}
+ \underbrace{\sum_{s'}T(s,a,s') \gamma V_\pi(s')}_{\text{(discntd) exp'd future rwd}}$
\underline{$V_\pi(s) = 0$ if $\text{IsEnd}(s)$};
otherwise a recurrence:
\begin{displaymath}
\boxed{
\begin{aligned}
    &V_\pi(s) = Q_\pi(s,\pi(s)) \\
    =&\sum_{s'}T(s,\pi(s),s')\left[\text{Reward}(s,\pi(s),s') + \gamma V_\pi(s')\right]
\end{aligned}
}
\end{displaymath}

\subsubsection{Algorithms}

\Red{Policy evaluation} iteratively computes $V_\pi$ (given \underline{a
specific} policy $\pi$).
\begin{itemize}
    \item \textbf{Initialization}: \fbox{$\forall s, V_\pi^{(0)}(s) \leftarrow 0$}
    \item \textbf{Iteration} $t = 1,\dots,T_{PE}$:\\
        $\forall s$ (for each state), \fbox{$V_\pi^{(t)}(s) \leftarrow Q_\pi^{(t-1)}(s,\pi(s))$}
        with \begin{displaymath}
            \boxed{
                \begin{aligned}
                    &Q_\pi^{(t-1)}(s,\pi(s)) = \sum_{s'}T(s,\pi(s),s')\\
                    &\left[\text{Reward}(s,\pi(s),s') + \gamma V_\pi^{(t-1)}(s')\right]
                \end{aligned}
            }
        \end{displaymath}
    \item until values don't change much: \fbox{$\max_{s \in S} |V_\pi^{(t)}(s) - V_\pi^{(t-1)}(s)| \le \epsilon$}
        (error tolerance)
\end{itemize}
Time complexity $O(T_{PE}|S||S'|)$ ($|S|$: \# of states; $|S'|$: \# of $s'$ with $T(s,a,s') > 0$)

\Blue{Optimal $Q$-value (of state $s$ with action $a$)} the maximum $Q$-value
attained by any policy taken after taking action $a$ from state $s$.
\begin{displaymath}
\boxed{
\begin{aligned}
    Q_{opt}(s,a) =& \sum_{s'}T(s,a,s')\\
    &\left[\text{Reward}(s,a,s') + \gamma V_{opt}(s')\right]
\end{aligned}
}
\end{displaymath}

\Blue{Optimal value (of state $s$)} the maximum value attained by any policy.
\underline{$V_{opt}(s) = 0$ if $\text{IsEnd}(s)$};
otherwise \fbox{$V_{opt}(s) = \max_{a \in A(s)} Q_{opt}(s,a)$}

\Blue{Optimal policy (of state $s$)} (``opt'' is still a policy!) the policy that leads
to the optimal values.
$\forall s$ \fbox{$\pi_{opt}(s) = \arg\max_{a \in A(s)} Q_{opt}(s,a)$}

\Red{Value iteration} finds $V_{opt}(s)$ and therefore $\pi_{opt}(s)$. \textbf{\underline{Off-policy}}
\begin{itemize}
    \item \textbf{Initialization}: \fbox{$\forall s, V_{opt}^{(0)}(s) \leftarrow 0$}
    \item \textbf{Iteration} $t = 1,\dots,T_{VI}$:\\
        $\forall s$ (for each state), \fbox{$V_{opt}^{(t)}(s) \leftarrow \max_{a \in A(s)} Q_{opt}^{t-1}(s,a)$}
        with \begin{displaymath}
            \boxed{
                \begin{aligned}
                    &Q_{opt}^{t-1}(s,a) = \sum_{s'}T(s,a,s')\\
                    &\left[\text{Reward}(s,a,s') + \gamma V_{opt}^{(t-1)}(s')\right]
                \end{aligned}
            }
        \end{displaymath}
    \item VI convergence guaranteed by
        \textbf{either $\gamma < 1$ or the MDP graph being acyclic}.
\end{itemize}
Time complexity $O(T_{VI}|S||A||S'|)$ ($|S|$: \# of states; $|A|$: \# of
actions per state; $|S'|$: \# of $s'$ with $T(s,a,s') > 0$)

\subsubsection{Reinforcement Learning}

Dealing with unknown $T$ and $\text{Rewards}$.

\Blue{MDPs (offline)} have a mental model of the world; find
$\pi_{opt}$ to maximize rewards collected.\\
\Blue{RL (online)} don't know how the world works;
perform actions to find out and collect rewards.

\Green{On-policy} estimate the value of a data-generating (exploration)
policy (usually to get $Q_{\pi}$).\\
\Green{Off-policy} estimate the value of another policy (usually to get
$Q_{opt}$).

\Red{Model-based Monte Carlo} estimates $T(s,a,s')$ and $\text{Reward}(s,a,s')$
using findings from \emph{randomly} traversing the space:
\fbox{$\hat{T}(s,a,s') = \frac{\text{\# times }(s,a,s')\text{ occurs}}{\text{\# times }(s,a)\text{ occurs}}$}
and \fbox{$\hat{\text{Reward}}(s,a,s') = r \text{ in } (s,a,r,s')$}.
These estimations can then be used to deduce $Q$-values ($\hat{Q}_\pi$ and $\hat{Q}_{opt}$).
\textbf{\underline{Off-policy}} because we explore the space arbitrarily to find
$\pi_{opt}$, independently from an exact policy.

\Red{Model-free Monte Carlo} directly estimates $Q_\pi$.
\fbox{$\hat{Q}_\pi(s,a) =$ average of $u_t$ where $s_{t-1} = s, a_t = a$}
($u_t$: the utility starting at step $t$ of a given episode).
\textbf{\underline{On-policy}} because the estimated value is dependent
on the policy $\pi$ used to generate the data.

\textbf{Convex combination formula} - for each $(s,a,u)$ of the training set and by
introducing \fbox{$\eta = \frac{1}{1 + (\text{\# updates to } (s,a))}$}:
\begin{displaymath}
\boxed{
\begin{aligned}
    \hat{Q}_\pi(s,a) \leftarrow & \hat{Q}_\pi(s,a) - \eta \left[
        \textcolor{red}{\underbrace{\hat{Q}_\pi(s,a)}_{\text{prediction}}}
        - \textcolor{Green}{\underbrace{u}_{\text{target}}}
    \right]
\end{aligned}
}
\end{displaymath}
Issue: reaching state-action pair $(s_{t-1}, a_t)$ required the sum until
termination ($u_t = \sum_{i=0}^{\infty}\gamma^i r_{t+1}$) just for a signle
update.

\Green{SARSA} estimate $Q_\pi$ by using both raw data (observed $r$) and
prediction (estimate of $\hat{Q}_\pi$) as part of the update rule. For each
tuple $(s,a,r,s',a')$ in the sequence of exploration via $\pi$
(\textbf{\underline{on-policy}}):
\begin{displaymath}
\boxed{
\begin{aligned}
    \hat{Q}_\pi(s,a) \leftarrow & \hat{Q}_\pi(s,a) \\
    &- \eta \left[
        \textcolor{red}{\underbrace{\hat{Q}_\pi(s,a)}_{\text{prediction}}}
        - \textcolor{Green}{\underbrace{r}_{\text{data}} + \gamma \underbrace{\hat{Q}_\pi(s',a')}_{\text{estimate}}}
    \right]
\end{aligned}
}
\end{displaymath}
SARSA estimate is updated on the fly as opposed to the model-free MC estimate
where the estimate can only be updated at the end of the episode. However if
rewards only exist at the terminal state, SARSA updates would be slower than
Model-free Monte Carlo.

\Blue{$Q$-learning} (\textbf{\underline{off-policy}}) produces an estimate for
$Q_{opt}$. For each $(s,a,r,s',a')$:
\begin{displaymath}
\boxed{
\begin{aligned}
    \hat{Q}_{opt}(s,a) \leftarrow \hat{Q}_{opt}(s,a) \\
    - \eta \left[
        \textcolor{red}{\underbrace{\hat{Q}_{opt}(s,a)}_{\text{prediction}}}
        - \textcolor{Green}{\underbrace{(r + \gamma \max_{a' \in A(s')}\hat{Q}_{opt}(s',a'))}_{\text{target}}}
    \right]
\end{aligned}
}
\end{displaymath}

\Green{Exploration vs Exploitation}
\begin{itemize}
    \item Too greedy (always picking the best action): won't explore everywhere
    \item Too much exploring: learn too slowly
\end{itemize}

\Green{Epsilon-greedy policy} an algorithm that balances exploration with
probability $\epsilon$ and exploration with probability $1-\epsilon$. For a
given state $s$, the policy is computed as:
\fbox{
    $\pi_{act}(s) = \begin{cases}
        \arg\max_{a \in A(s)} \hat{Q}_{opt}(s,a) & 1-\epsilon\\
        \text{random action from } A(s) & \epsilon
    \end{cases}$
}

\Blue{$Q$-learning with function approximation} avoid memorizing every state as
$\hat{Q}_{opt}(s,a)$ just maps $(s,a)$ to a value estimate so define
$\hat{Q}_{opt}(s,a; \mathbf{w}) = \mathbf{w} \cdot \phi(s,a)$ where the learned
$\mathbf{w}$ can replace the mapping table.
For each $(s,a,r,s',a')$:
\begin{displaymath}
\boxed{
\begin{aligned}
    \mathbf{w} \leftarrow \mathbf{w} - \eta \\
    \left[
        \textcolor{red}{\underbrace{\hat{Q}_{opt}(s,a; \mathbf{w})}_{\text{prediction}}}
        - \textcolor{Green}{\underbrace{(r + \gamma \max_{a' \in A(s')}\hat{Q}_{opt}(s',a'; \mathbf{w}))}_{\text{target}}}
    \right] \\
    \phi(s,a)
\end{aligned}
}
\end{displaymath}