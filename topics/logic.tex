\subsection{Logic Basics}

\Blue{Syntax} representation of a meaning: what are valid expressions in the language?

\Blue{Semantics} what do these expressions mean?

\Green{Inference rules} what does a formula imply?

\Blue{Propositional symbols} (atomic formulas): $A$, $B$, $C$

\Red{Model $w$} denotes an \underline{assignment} of truth values to
propositional symbols.
\Hint{Example: 3 propositional symbols yield $2^3=8$ possible models.}

\Blue{Interpretation function $\mathcal{I}(f,w)$} outputs whether model $w$
satisfies formula $f$: \fbox{$\mathcal{I}(f,w) = \begin{cases}
    1 & w \text{ satisfies } f\\
    0 & \text{otherwise}
\end{cases}$}

\Green{Set of models $\mathcal{M}(f)$} denotes the set of models $w$ that
satisfy formula $f$:
\fbox{$\forall w \in \mathcal{M}(f)$, $\mathcal{I}(f,w) = 1$}

\Green{Knowledge base $KB$} a set of formulas representing their conjunction:
\fbox{$\mathcal{M}(KB) = \cap_{f \in KB} \mathcal{M}(f)$}
$KB$ specifies constraints on the world. $\mathcal{M}(KB)$ is the set of all
worlds satisfying those constraints.
\Hint{Adding more formulas to the KB shrinks the set of models (more
constrained).}

\Red{Entailment $KB \models f$} KB entails $f$ iff $\mathcal{M}(KB) \subseteq
\mathcal{M}(f)$ \Hint{$f$ doesn't bring any new info.}

\Red{Contradiction $KB \models \neg f$} KB contradicts $f$ iff $\mathcal{M}(KB) \cap \mathcal{M}(f)
= \emptyset$
\Hint{No model satisfies the constraints after adding $f$.}

\Red{Contigency} $\emptyset \subsetneq \mathcal{M}(KB) \cap \mathcal{M}(f) \subsetneq \mathcal{M}(KB)$.
\Hint{$f$ adds a non-trivial amount of info to KB and doesn't contradict KB.}

\Blue{Probabilistic interpretation} the probability that query $f$ is evaluated
to $1$ can be seen as the proportion of models $w$ of the KB that satisfy $f$:
\fbox{
    $P(f|KB) = \frac
        {\sum_{w \in \mathcal{M}(KB) \cap \mathcal{M}(f)} P(W=w)}
        {\sum_{w \in \mathcal{M}(KB)} P(W=w)}$
}

\Green{Satisfiability} the KB is said to be satisfiable if at least one model
$w$ satisfies all its constraints, i.e.
\fbox{$\mathcal{M}(KB) \neq \emptyset$}.
\begin{itemize}
    \item Is $KB \cup \{\neg f\}$ satisfiable? \begin{itemize}
        \item no: \textbf{entailment}
        \item yes: Is $KB \cup \{f\}$ satisfiable? \begin{itemize}
            \item no: \textbf{contradiction}
            \item yes: \textbf{contingent}
        \end{itemize}
    \end{itemize}
\end{itemize}

\Blue{Model checking} input: a KB and output: whether the KB is satisfiable or
not.

\begin{tabular}{|c|c|}
    \hline
    \textbf{Model checking} & \textbf{CSP} \\
    \hline
    propositional symbol & variable \\ 
    \hline
    formula & constraint \\
    \hline
    model & assignment\\
    \hline
\end{tabular}

\Red{Inference rule} of premises $f_1, \dots, f_k$ and conclusion $g$ is written:
\fbox{$\frac{f_1, \dots, f_k}{g}$} where $f$s and $g$ are formulas.
\Hint{Rules operate directly on syntax, not on semantics.}

\Green{Modus ponens inference rule} for any propositional symbols $p$ and $q$:
\fbox{$\frac{p, p \rightarrow q}{q}$}.
Example: $\frac{\text{Rain}, \text{Rain} \rightarrow \text{Wet}}{\text{Wet}}$
\begin{itemize}
    \item Premises\begin{itemize}
        \item It's raining
        \item If it's raining, then it's wet
    \end{itemize}
    \item Conclusion: therefore, it's wet
\end{itemize}

\Red{Forward inference} a model checking algorithm. Input: set of inference
rules Rules. Repeat until no changes to KB:
\begin{itemize}
    \item Choose set of formulas $f_1, \dots, f_k \in KB$
    \item If matching rule $\frac{f_1, \dots, f_k}{g}$ exists:\begin{itemize}
        \item Add $g$ to $KB$
    \end{itemize}
\end{itemize}

\Green{Derivation $KB \vdash f$} $KB$ \textbf{derives/proves} $f$ iff $f$
eventually gets added to $KB$.

\Blue{Soundness} a set of inference rules Rules is sound if:
\fbox{$\left\{f:KB \vdash f\right\} \subseteq \left\{f:KB \models f\right\}$}.
\begin{itemize}
    \item Inferred formulas are entailed by $KB$
    \item Can be checked one rule at a time
    \item \Hint{\emph{Nothing but the truth}}
\end{itemize}
Examples:
Modus ponens such as
$\frac{\text{Rain}, \text{Rain} \rightarrow \text{Wet}}{\text{Wet}}$ is sound.
$\frac{\text{Rain}, \text{Rain} \rightarrow \text{Wet}}{\text{Rain}}$ is not sound.

\Blue{Completeness} a set of inference rules Rules is complete if:
\fbox{$\left\{f:KB \vdash f\right\} \supseteq \left\{f:KB \models f\right\}$}.
\begin{itemize}
    \item Formulas entailing $KB$ are either already in the knowledge base or inferred from it
    \item \Hint{\emph{The whole truth}}
\end{itemize}
Modus ponens is incomplete.

\subsubsection{Propositional logic}

\Red{Horn clause} is either:
\begin{itemize}
    \item a \underline{definite clause}: \fbox{$(p_1 \wedge \dots \wedge p_k) \rightarrow q$}
    \item a \underline{goal clause}: \fbox{$(p_1 \wedge \dots \wedge p_k) \rightarrow \text{false}$}
        \Hint{Interpreted as the negation of the conjunction.}
\end{itemize}

\Red{Modus ponens inference rule}
\fbox{$\frac{p_1, \dots, p_k, (p_1 \wedge \dots \wedge p_k) \rightarrow q}{q}$}
Modus ponens is \textbf{complete} w.r.t. Horn clauses:
\begin{itemize}
    \item if we suppose that $KB$ contains only Horn clauses and $q$ is an entailed propositional symbol.
    \item then applying modus ponens will derive $q$.
\end{itemize}

\Green{Conjunctive normal form (CNF)} $\wedge$ (and) or $\vee$ (or-s).
\Hint{Equivalent: KB where each formula is a clause.}

\Red{Convert into CNFs}
\begin{tabular}{c|l}
    \hline
    $f \rightarrow g$ & $\neg f \vee g$ \\ 
    \hline
    $f \leftrightarrow g$ & $(f \rightarrow g) \wedge (g \rightarrow f)$ \\ 
    \hline
    $\neg (f \wedge g)$ & $\neg f \vee \neg g$\\
    \hline
    $\neg (f \vee g)$ & $\neg f \wedge \neg g$\\
    \hline
    $f \vee (g \wedge h)$ & $(f \vee g) \wedge (f \vee h)$\\
    \hline
\end{tabular}

\Hint{\textbf{Use parentheses around $(f \rightarrow g)$ to ``beat'' $\wedge$}}

\Green{Resolution inference rule} \fbox{
    $\frac{f_1 \vee \dots \vee f_k \vee p, \neg p \vee g_1 \vee \dots \vee g_m}
    {f_1 \vee \dots \vee f_k \vee g_1 \vee \dots \vee g_m}$
}
Example: $\frac{\text{Rain} \vee \text{Snow}, \neg \text{Snow} \vee \text{Traffic}}
{\text{Rain} \vee \text{Traffic}}$

\Blue{Resolution-based inference} recall $KB \models f \leftrightarrow KB \cup \{\neg f\}$ is unsatisfiable.
\begin{itemize}
    \item Add $\neg f$ to $KB$
    \item Convert all formulas into CNF.
    \item Repeatedly apply resolution rule.
    \item Return entailment iff False is derived.
\end{itemize}

\subsubsection{First-order logic}

\Blue{Terms} expressions refer to objects:
\begin{itemize}
    \item Constant symbol: Alice
    \item Variable: $x$
    \item Function of terms: $sum(x, 3)$
\end{itemize}

\Blue{Formulas} refer to truth values:
\begin{itemize}
    \item Atomic formula (predicate applied to terms): Knows(Bob, Arith)
    \item Connectives applied to formulas
    \item Quantifiers applied to formulas
\end{itemize}

\begin{itemize}
    \item \Hint{see ``every'' thinks $\forall$ and $\rightarrow$}: \begin{itemize}
        \item Every student knows arithmetic: $\forall x \text{Student}(x) \rightarrow \text{Knows}(x, \text{arithmetic})$
    \end{itemize}
    \item \Hint{see ``some'' thinks $\exists$ and $\wedge$}: \begin{itemize}
        \item Some student knows arithmetic: $\exists x \text{Student}(x) \wedge \text{Knows}(x, \text{arithmetic})$
    \end{itemize}
    \item $\neg \forall x P(x)$ equivalent to $\exists x \neg P(x)$
\end{itemize}

\Green{Model $w$} in first-order logic maps:
\begin{itemize}
    \item constant symbols to objects: e.g. $w(\text{alice}) = o_1, w(\text{bob}) = o_2, w(\text{arithmetic}) = o_3$
    \item predicate symbols to tuple of objects: e.g. $w(\text{Knows}) = \{(o_1, o_3), (o_2, o_3), \dots\}$
\end{itemize}

Two assumptions to simplify things:
\begin{itemize}
    \item \Hint{Unique names assumption}: each object has \textbf{at most one} constant symbol.
    \item \Hint{Domain closure}: each object has \textbf{at least one} constant symbol.
\end{itemize}
\Hint{\textbf{Together}}: 1-to-1 mapping between constant symbols in syntax-land
and objects in semantics-land.\\
Then: First-order logic is syntactic sugar for propositional logic $\rightarrow$
\Hint{use any inference algorithm for propositional logic.}
\Blue{Propositionalization} example:
\begin{tabular}{l|l}
    \hline
    $\text{Stu}(\text{A}) \wedge \text{Stu}(\text{A})$ & $\text{StuB} \wedge \text{StuB}$ \\ 
    \hline
    $\forall x \text{Stu}(x) \rightarrow \text{Per}(x)$ & $(\text{StuA} \rightarrow \text{PerA})$\\
    & $\wedge (\text{StuB} \rightarrow \text{PerB})$ \\
    \hline
    $\exists x \text{Stu}(x) \wedge \text{Creative}(x)$ & $(\text{StuA} \wedge \text{CreativeA})$\\
    & $\vee (\text{StuB} \rightarrow \text{CreativeB})$ \\
    \hline
\end{tabular}

\Red{Horn clause}
\fbox{$\forall x_1, \dots, \forall x_n, (a_1 \wedge \dots \wedge a_k) \rightarrow b$}
$x_1, \dots x_n$ are variables and $a_1, \dots a_k, b$ are atomic formulas which
contain those variables.

\Green{Substitution} a substitution $\theta$ maps variables to terms and
$\text{Subst}[\theta, f]$ denotes the result of substitution $\theta$ on $f$.

\Green{Unification} takes two formulas $f$ and $g$ and returns a substitution $\theta$
which is the most general substitution $\theta$ that makes them equal:
\fbox{$\text{Unify}[f,g] = \theta$ such that $\text{Subst}[\theta, f] = \text{Subst}[\theta, g]$}.
$\text{Unify}[f,g]$ returns Fail if no such $\theta$ exists.

\Blue{Modus ponens} by calling
$\theta = \text{Unify}[\textcolor{blue}{a'_1 \wedge \dots \wedge a'_k}, \textcolor{red}{a_1 \wedge \dots \wedge a_k}]$,
\fbox{
    $\frac{\textcolor{blue}{a'_1 \wedge \dots \wedge a'_k}, \forall x_1, \dots, \forall x_n (\textcolor{red}{a_1 \wedge \dots \wedge a_k}) \rightarrow b}
    {\text{Subst}[\theta, b]}$
}

\Red{Completeness} Modus ponens is complete for first-order logic with only Horn clauses.
\Blue{semi-decidability} first-order logic (even restricted to only Horn
clauses) is semi-deciable.
\begin{itemize}
    \item if $KB \models f$, forward inference on complete inference rules will prove $f$ in finite time
    \item if $KB \nrightmodels f$, no algorithm can show this in finite time
\end{itemize}

\Red{Resolution rule} by calling $\theta = \text{Unify}[p,q]$,
\fbox{
    $\frac{\textcolor{blue}{f_1 \vee \dots \vee f_k} \vee p, \neg q \vee \textcolor{red}{g_1 \vee \dots \vee g_m}}
    {\text{Subst}[\theta, \textcolor{blue}{f_1 \vee \dots \vee f_k} \vee \textcolor{red}{g_1 \vee \dots \vee g_m}]}$
}